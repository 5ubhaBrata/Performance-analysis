\documentclass[12pt]{article}
\usepackage[margin=30 mm]{geometry}
\usepackage{mathrsfs}
\usepackage{amssymb}
\usepackage{graphicx}
\usepackage{amsmath}
\usepackage{paralist}
\usepackage{graphicx,wrapfig,lipsum}
\title{} 
\author{} 
\date{}
\begin{document}
\begin{titlepage} 
	\thispagestyle{empty}
	
	
	\maketitle 
	\centerline{\begin{large}
			\textbf{STUDY THE EFFECTS OF NON-SKILLED VARIABLES }
	\end{large}}
	\centerline{{\begin{large}
				 \textbf{ON THE PERFORMANCE OF FOOTBALL STRIKERS}  
	\end{large}} }

	\begin{center}
		
		SUBMITTED BY: SUBHABRATA ADAK
		
		ENROLMENT ID: CUSB2002212009
		
		M.SC. IN STATISTICS
		
		SESSION: 2020 - 2022
		
		CENTRAL UNIVERSITY OF SOUTH BIHAR
		
		UNDER SUPERVISION OF 
		
		DR. KAMLESH KUMAR
		
		ASSISTANT PROFESSOR, CUSB
	\end{center}
	
\end{titlepage} 
\newpage


\section*{\underline{Abstruct}:}

Football is one of the most popular sports around the globe; and club football obviously has a great role on it. Although most of the countries enroled with the game but European \textquotedblleft big five\textquotedblright  countries are dominating the world football right now. Which are obviously England, Germany, Spain, Italy and France. 

The idea of this project is to find whether there any relationship between performance of football forwards and different variables; and which variable impact how much on their performance using statistical methods.

\section{\underline{Introduction}:}
\subsection{Background:}

Football (or soccer as the game is called in some parts of the world) has a long history. Football in its current form arose in England in the middle of the 19th century. But alternative versions of the game existed much earlier and are a part of the football history.

 Since then it has eveloped a lot. Now a days it is one of the most popular sports, not only sports it’s an emotion for enthusias, around the globe. There is a great role of club football behind the popularity of the game. Mainly, because fans get to watch their favourite players from different countries played in same team and because of the craze sponsor find a very profitable market to invest in.
 
  Although almost every country that played football has its own league, but when we talk about club football Europe is like heaven of it both compitition and revenue wise. And England, France, Spain, Germany and Italy are the giants. 
  
  \begin{wrapfigure}{l}{11cm}
  	\caption{ The giants of football }\label{wrap-fig:1}
  	\includegraphics[width=10 cm]{Screenshot at 2021-12-24 12-54-38.png}
  \end{wrapfigure} 
  
  
  Not only club football in world cup also, there are 21 football world cup till date among which 12 are won by these five european giants. Only other countries that have won a world cup are Brazil, Argentina and Uruguay. 
  
  These are the countries to host top five major leagues all over the world. Which are La Liga in Spain, Premier League in England, Serie A in Italy, Ligue 1 in France and Bundesliga in Germany. 
  
  Here are some reason why these are the major leagues around the world;
  
   Almost every big name in football history is playing or has played in at least one of these leagues.
   
\begin{wrapfigure}{r}{11cm}
	\caption{Brand value of football clubs by country }\label{wrap-fig:1}
	\includegraphics[width=11cm]{Screenshot 2021-11-30 at 20-38-55 Football Annual 2020 Brand Value Analysis Charts Brandirectory.png}
\end{wrapfigure} 
Here is the annual report on the most valuable and strongest football brands (clubs) (BrandFinance, 2020). According to this report these countries, clearly, has almost full control over the world of football. \\


\begin{wrapfigure}{l}{7.5cm}
	\caption{Average attendence per match}\label{wrap-fig:1}
	\includegraphics[width=7.5cm]{9-38.jpg}
\end{wrapfigure}
 
A report of insidersports, June 26, 2018, shows “Top 10: Most watched football leagues in the world” according to avarage attendance (shows in the Figure 3). We can see that, Germany, England, Spain, Italy tops the list and France is in the 8th place. 
\subsection{Literature Review}
There are plenty of work which are tried to measure the performance of players using variables which are not consider as "skill" of a player. Here are some of them:

\textbf{ (1) } In a review article, Vanessa Martínez-Lagunas, Margot Niessenand Ulrich Hartmann(2014) have studied the specific characteristics of female football players and the demands of match-play. In this article, mean values reported in the literature for age (12-27 years),body height (155-174 cm), body mass (48-72 kg), percent body fat (13\%-29\%), maximal oxygen uptake (45.1-55.5 mL/kg/min), Yo-YoIntermittent Recovery Test Level 1 (780-1379 m), maximum heart rate (189-202 bpm), 30 m sprint times (4.34-4.96 s), and counter-movement jump or vertical jump (28-50 cm) vary mostly according to the players’ competitive level and positional role.
	 
\textbf{ (2)  } In an article, George P. Nassis tries to find if there is  any effect of altitude on football performance or not using the data of 2010 world cup. The main finding of his study was that the teams’ endurance performance, determined by the total distance covered during the game, was 3.1\% lower in the matches played at altitudes above 1,200 m during the 2010 World Cup compared with sea-level values. However, it is noteworthy that the maximal speed, the number of goals scored, and the errors made by the goalkeepers that resulted in goals conceded were not significantly influenced by altitude.
	 
\textbf{(3) } In his article John Bica shows that during covid-19 pandamic absence of the fans has a effect on performence of the team. They have won more and lose less, reletively speaking, at home ground when fans are present at the stadium.


\subsection{Problem Statement}
A footballer’s performence obviously depend on his/her skill. But, we can not measure one’s skill, atmost we can rate them. So, if we tried to measure one player’s performance we have to look for different ideas. As we have seen there are some work which are tried to measure the performance of players using variables which are does not consider as \textquotedblleft skill\textquotedblright of a player. My objective of this project is to do something like that. Hence the title “Non-skilled Variable”.
\subsubsection{Objective}
A footballer’s performence obviously depend on his/her skill. But, we can not measure one’s skill, atmost we can rate them. So, if we tried to measure one player’s performance we have to look for different ideas. As we have seen there are some work which are tried to measure the performance of players using variables which are does not consider as "skill" of a player. our aim in this project is to do something like that. Hence the title “Non-skilled” Variable.

Our objective of this project is to find whether there any relationship between performance of football forwards(strikers) and “non-skilled” variables. If there is, then we study which variable impact how much on the performance. If there isn’t, then we will try to find the possible reason.


\subsubsection{Data}
In this project we are going to consider the data for 2019/20 and 2020/21 season of football strikers from major five european league i.e, La Liga, Premier League, Serie A, Ligue 1 and Bundesliga. I have mainly collect the data from three websites; bdfutball, whoscored, tranfermarkt, and cross checked the data with official websites of each leagues. 

\subsubsection*{Variables:}
We may measure strikers’ performance by their number of goals commited, number of assists with goal, number of shots taken, number of duel won etc. The variables which can possibly effect the performance, non-skilled variable, are a players age, height, weight, BMI, his nationality, in which club, under which manager he playes for. How many minutes played by him, in how many matches he was included in the starting $ XI $, how many fouls commited how many foul drawn, how many red crad and yellow card he has seen, how many offside commited, how many matches missed due to injuries.

\subsubsection*{Selection of Variables:}
\begin{itemize}
	\item As we want to measure the performance of football forwards it is kind of obvious to choose number of goals and number of assists with goal as dependent variable.
	\item Other than goals and assists their could be many variable which represents a strikers' performance. Number of shots taken per game could be a key point here.
	\item Football is not only about taking shots and scoring goals, you have to control the ball and keep the pressure on opponent in order to win the game. A airal dual won could be consider here. 

\end{itemize}





\end{document}